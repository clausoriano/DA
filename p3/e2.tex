void ordenacionInsercion(Candidates& candidatos, int n)
{
 	int i, j; 
	Cell temp;
 	for(i=0; i<n; i++)
 	{
 		temp=candidatos[i];
 		j=i-1;
 		while(j>=0 && candidatos[j].value >temp.value)
 		{
 			candidatos[j+1] = candidatos[j];
 			j--;
 		}

		candidatos[j+1] = temp;
	}
}

void fusion(Candidates& candidatos, int l, int m, int r)
{
	int i, j , k;
	int n1 = m - l  + 1;
	int n2 = r - m;

	Cell L[n1], R[n2];

	for (i = 0; i < n1; i++) 
	{
        L[i] = candidatos[l + i]; 
	}
    for (j = 0; j < n2; j++)
	{ 
        R[j] = candidatos[m + 1+ j];
	} 

	i = 0;
	j = 0;
	k = l;

	while (i < n1 && j < n2) 
    { 
        if(L[i].value > R[j].value) 
        { 
            candidatos[k] = L[i]; 
            i++; 
        } 
        else
        { 
            candidatos[k] = R[j]; 
            j++; 
        } 
        k++; 
    }

	while(i < n1) 
    { 
        candidatos[k] = L[i]; 
        i++; 
        k++; 
    }  

	 while(j < n2) 
    { 
        candidatos[k] = R[j]; 
        j++; 
        k++; 
    } 
}

//Algoritmo CON PREORDENACION - FUSION
void ordenacionFusion(Candidates& candidatos, int inicio, int fin)
{
	int n = fin - inicio + 1;
	if (n==0 || n==1)
	{
		ordenacionInsercion(candidatos, n);
	}
	else
	{
		int k = inicio - 1 + (n/2);
		ordenacionFusion(candidatos, inicio, k);
		ordenacionFusion(candidatos, k+1, fin);
		fusion(candidatos, inicio, k, fin);
	}
	

}
