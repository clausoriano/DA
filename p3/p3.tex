\documentclass[]{article}

\usepackage[left=2.00cm, right=2.00cm, top=2.00cm, bottom=2.00cm]{geometry}
\usepackage[spanish,es-noshorthands]{babel}
\usepackage[utf8]{inputenc} % para tildes y ñ
\usepackage{graphicx} % para las figuras
\usepackage{xcolor}
\usepackage{listings} % para el código fuente en c++

\lstdefinestyle{customc}{
  belowcaptionskip=1\baselineskip,
  breaklines=true,
  frame=single,
  xleftmargin=\parindent,
  language=C++,
  showstringspaces=false,
  basicstyle=\footnotesize\ttfamily,
  keywordstyle=\bfseries\color{green!40!black},
  commentstyle=\itshape\color{gray!40!gray},
  identifierstyle=\color{black},
  stringstyle=\color{orange},
}
\lstset{style=customc}


%opening
\title{Práctica 3. Divide y vencerás}
\author{\input{../autor}}


\begin{document}

\maketitle

%\begin{abstract}
%\end{abstract}

% Ejemplo de ecuación a trozos
%
%$f(i,j)=\left\{ 
%  \begin{array}{lcr}
%      i + j & si & i < j \\ % caso 1
%      i + 7 & si & i = 1 \\ % caso 2
%      2 & si & i \geq j     % caso 3
%  \end{array}
%\right.$

\begin{enumerate}
\item Describa las estructuras de datos utilizados en cada caso para la representación del terreno de batalla. 

$$ f(radio,rango,salud)=(radio+rango)-1+rango^2+3\frac{radio}{salud} $$

Escriba aquí su respuesta al ejercicio 1.

\item Implemente su propia versión del algoritmo de ordenación por fusión. Muestre a continuación el código fuente relevante. 

void ordenacionInsercion(Candidates& candidatos, int n)
{
 	int i, j; 
	Cell temp;
 	for(i=0; i<n; i++)
 	{
 		temp=candidatos[i];
 		j=i-1;
 		while(j>=0 && candidatos[j].value >temp.value)
 		{
 			candidatos[j+1] = candidatos[j];
 			j--;
 		}

		candidatos[j+1] = temp;
	}
}

void fusion(Candidates& candidatos, int l, int m, int r)
{
	int i, j , k;
	int n1 = m - l  + 1;
	int n2 = r - m;

	Cell L[n1], R[n2];

	for (i = 0; i < n1; i++) 
	{
        L[i] = candidatos[l + i]; 
	}
    for (j = 0; j < n2; j++)
	{ 
        R[j] = candidatos[m + 1+ j];
	} 

	i = 0;
	j = 0;
	k = l;

	while (i < n1 && j < n2) 
    { 
        if(L[i].value > R[j].value) 
        { 
            candidatos[k] = L[i]; 
            i++; 
        } 
        else
        { 
            candidatos[k] = R[j]; 
            j++; 
        } 
        k++; 
    }

	while(i < n1) 
    { 
        candidatos[k] = L[i]; 
        i++; 
        k++; 
    }  

	 while(j < n2) 
    { 
        candidatos[k] = R[j]; 
        j++; 
        k++; 
    } 
}

//Algoritmo CON PREORDENACION - FUSION
void ordenacionFusion(Candidates& candidatos, int inicio, int fin)
{
	int n = fin - inicio + 1;
	if (n==0 || n==1)
	{
		ordenacionInsercion(candidatos, n);
	}
	else
	{
		int k = inicio - 1 + (n/2);
		ordenacionFusion(candidatos, inicio, k);
		ordenacionFusion(candidatos, k+1, fin);
		fusion(candidatos, inicio, k, fin);
	}
	

}



\item Implemente su propia versión del algoritmo de ordenación rápida. Muestre a continuación el código fuente relevante. 

void ordenacionInsercion(Candidates& candidatos, int n)
{
 	int i, j; 
	Cell temp;
 	for(i=0; i<n; i++)
 	{
 		temp=candidatos[i];
 		j=i-1;
 		while(j>=0 && candidatos[j].value >temp.value)
 		{
 			candidatos[j+1] = candidatos[j];
 			j--;
 		}

		candidatos[j+1] = temp;
	}
}

int pivote(Candidates& candidatos, int izq, int der)
{
	int i;
	int pivote;
	double valor_pivote;
    Cell aux;

    pivote = izq;
    valor_pivote = candidatos[pivote].value;
    for (i=izq+1; i<=der; i++){
        if (candidatos[i].value >= valor_pivote){
                pivote++;
                aux=candidatos[i];
                candidatos[i]=candidatos[pivote];
                candidatos[pivote]=aux;

        }
    }
    aux=candidatos[izq];
    candidatos[izq]=candidatos[pivote];
    candidatos[pivote]=aux;
    return pivote;
}


//Algoritmo CON PREORDENACION - RAPIDA
void ordenacionRapida(Candidates& candidatos, int inicio, int fin )
{
	int n = fin - inicio + 1;
	if (n==0 || n==1)
	{
		ordenacionInsercion(candidatos, n);
	}
	else
	{
		int p = pivote(candidatos, inicio, fin);
		ordenacionRapida(candidatos, inicio, p-1);
		ordenacionRapida(candidatos, p+1, fin);
	}
	
}


\item Realice pruebas de caja negra para asegurar el correcto funcionamiento de los algoritmos de ordenación implementados en los ejercicios anteriores. Detalle a continuación el código relevante.

Conjunto de Candidatos: candidatosCER, un vector de Cell que se ordenan de mayor a menor por su valor, asi todas las extracciones del vector seran de coste elemental.

Conjunto de Candidatos Seleccionados: Cell possible, que contiene al candidato seleccionado en cada iteración del algoritmo.

Funcion de Solucion: variable solution, es un bool que indica si se ha conseguido colocar el centro de recursos.

Funcion de Factibilidad: funfact(), comprueba que no se salga del mapa y no choque con obstaculos en el mapa.

Objetivo: elegir la posicion que tenga maxima puntuacion para colocar el centro de recursos.




\item Analice de forma teórica la complejidad de las diferentes versiones del algoritmo de colocación de defensas en función de la estructura de representación del terreno de batalla elegida. Comente a continuación los resultados. Suponga un terreno de batalla cuadrado en todos los casos. 

En el caso de no usar ningún tipo de preordenación, el orden del algoritmo es O(n²)

El orden del algoritmo de fusion es O(nlogn)

El orden del algoritmo de ordenación rapida es O(n²)

El orden del algoritmo con monticulo es O(nlogn)

Dependiendo de que algoritmo de ordenacion se utilice, cambiará el orden de nuestro algoritmo. Siendo O(nlogn) el más eficiente que podemos obtener. 


\item Incluya a continuación una gráfica con los resultados obtenidos. Utilice un esquema indirecto de medida (considere un error absoluto de valor 0.01 y un error relativo de valor 0.001). Considere en su análisis los planetas con códigos 1500, 2500, 3500,..., 10500. Incluya en el análisis los planetas que considere oportunos para mostrar información relevante.

Escriba aquí su respuesta al ejercicio 6.

\end{enumerate}

Todo el material incluido en esta memoria y en los ficheros asociados es de mi autoría o ha sido facilitado por los profesores de la asignatura. Haciendo entrega de este documento confirmo que he leído la normativa de la asignatura, incluido el punto que respecta al uso de material no original.

\end{document}
